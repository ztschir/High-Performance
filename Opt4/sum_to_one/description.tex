\documentclass[10pt]{article}

\usepackage{psfig}
\usepackage{moreverb}
\usepackage{latexsym}
\usepackage{rotating}
\usepackage{makeidx}  % allows for indexgeneration

\setlength{\textheight}{8.75in}
\setlength{\textwidth}{6.5in}
\setlength{\oddsidemargin}{0in}
\setlength{\evensidemargin}{0in}
\setlength{\topmargin}{-0.75in}

\newtheorem{theorem}{Theorem}[section]
\newtheorem{conjecture}[theorem]{Conjecture}
\newtheorem{corollary}[theorem]{Corollary}
\newtheorem{proposition}[theorem]{Proposition}
\newtheorem{lemma}[theorem]{Lemma}
\newtheorem{definition}[theorem]{Definition}
\newtheorem{remark}[theorem]{Remark}
\newtheorem{exercise}[theorem]{Exercise}
\newtheorem{example}[theorem]{Example}
\newtheorem{note}{Note}
\newtheorem{algorithm}{Algorithm}

% \newcommand{\PPre}{P_{\rm pre}}
% \newcommand{\PPost}{P_{\rm post}}
% \newcommand{\PInv}{P_{\rm inv}}

\newenvironment{indent0}%
%  {\begin{list}{\  \hspace{.0in} }{}}%
%  {\end{list}}
  { \begin{description} }
  { \end{description} }

\newenvironment{steps}
{ \begin{list}{\bf Step ??:}{}}
{ \end{list} }

\newenvironment{indent1}%
  {\begin{list}{\  \hspace{.05in} }{}}%
  {\end{list}}

\newenvironment{indent2}%
  {\begin{list}{\  \hspace{.1in} }{}}%
  {\end{list}}

\newenvironment{indent3}%
  {\begin{list}{\  \hspace{.15in} }{}}%
  {\end{list}}

\newcommand{\undefined}{\star}

\title{
Assignment: Implementing the Reduce operation
}
            
\author{
Robert A. van de Geijn
}

\date{
}
\begin{document}

\maketitle

\begin{enumerate}
\item
On the CS machines, copy the contents of {\tt ~rvdg/class/CS378S13/mpi/sum\_to\_one}
to your directory.  
\begin{quote}
({\tt cp -r ~rvdg/class/CS378S13/mpi/sum\_to\_one <directory name>})
\end{quote}
\item
Change to that directory.
\item
Modify the file ``hostfile'' so that the machine that you are on is listed instad of {\tt diligence.cs.utexas.edu}.  What this will do is create multiple MPI processes, but they will all run on the same machine.  As a result, you will be simulating a distributed memory parallel computer on a single CPU.
\item
Type ``source setup'' or ``source setup\_bash''.  This initializes
some environment variables for MPI.
\item
Type ``make run'' //
This will compile the code and run it on 5 machines as a parallel computer.
\item
What does this code do?
\begin{enumerate}
\item
It tests the routine {\tt my\_sum\_to\_one} in file {\tt my\_sum\_to\_one.c}.
\item
It does so by using the Message-Passing Interface.  To learn more
about MPI, visit
{\tt http://www-unix.mcs.anl.gov/mpi/}.
\item
Let the processors be indexed $ 0, 1, 2, ... $.
Then on processor $ i $, the input vector is
\[
\left( \begin{array}{c}
0 + i * 10^{-(i+1)} \\
1 + i * 10^{-(i+1)} \\
\vdots \\
9 + i * 10^{-(i+1)}
\end{array}
\right)
\]
In other words, the input vectors are given by
\[
\left( \begin{array}{r}
0.0000 \\
1.0000 \\
\vdots
9.0000
\end{array}
\right), \quad
%
\left( \begin{array}{r}
0.0100 \\
1.0100 \\
\vdots
9.0100
\end{array}
\right),  \quad
%
\left( \begin{array}{r}
0.0020 \\
1.0020 \\
\vdots
9.0020
\end{array}
\right),  \quad
\mbox{etc.}
\]
on processors $ 0, 1, 2, \ldots $.
\item
The output is the sum of these vectors, gather to the root (in this
case processor 0):
\[
\left( \begin{array}{c}
0.01234 \\
5.01234 \\
\vdots \\
45.01234
\end{array}
\right)
\]
\end{enumerate}
\item
Notice that in the file {\tt my\_sum\_to\_one.c} I have merely
implemented the parallel summation of these vectors by a call
to the MPI library routine {\tt MPI\_Reduce}.
\item
Your job is to reimplement this using a minimum-spanning-tree
(MST) algorithm, as discussed in class.
\item
Hint: in {\tt my\_bcast.c} you will find a recursively implemented
MST broadcast, which you should be able to modify to come up
with a MST reduce.
\end{enumerate}
\end{document}

