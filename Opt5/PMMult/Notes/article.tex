\documentclass[10pt]{article}

\setlength{\textheight}{8.75in}
\setlength{\textwidth}{6.5in}
\setlength{\oddsidemargin}{0in}
\setlength{\evensidemargin}{0in}
\setlength{\topmargin}{-0.75in}

\usepackage{framed}
\usepackage{multirow}


\usepackage{epsfig}
\usepackage{amssymb}
\usepackage[normalem]{ulem}

\usepackage{amsthm}
\usepackage{program}


\usepackage{ifthen}
\usepackage{color}
\usepackage{boxedminipage}
\usepackage{fancyvrb}

\usepackage[dotinlabels]{titletoc}
\usepackage{colortbl}
\usepackage{array}
\usepackage{soul}
\usepackage{graphicx}
\usepackage{graphics}
\usepackage{moreverb}
\usepackage{rotating}
\usepackage{enumerate}
\usepackage{rotating}

\newcommand{\striu}{{\rm striu}}
\newcommand{\half}{\frac{1}{2}}


\newcommand{\hous}{{\rm Hous}}

\newcommand{\stack}[2]{
\mathord{
  \raise 1.2ex\hbox to 0pt 
          {$\scriptstyle#2$\hss}
  #1}}

\newcommand{\rowv}[1]{
\stack{#1}{\rightarrow}
}

\newcommand{\rowdv}[1]{
\stack{#1}{=}
}

\newcommand{\colv}[1]{
\mathord{
  \raise 1.8ex\hbox to 0pt 
          {$\scriptstyle \downarrow$\hss}
  #1}}


\newcommand{\coldv}[1]{
\stack{#1}{\|}
}

\newcommand{\local}[1]{
\mbox{\rm loc}( #1 )
}
%\newcommand{\mod}{\bmod} 
\newcommand{\Akk}{ A_{kk} }
\newcommand{\Ajk}{ A_{jk} }
\newcommand{\Ajj}{ A_{jj} }
\newcommand{\Aki}{ A_{ki} }
\newcommand{\Aji}{ A_{ji} }
\newcommand{\Aij}{ A_{ij} }
\newcommand{\Aik}{ A_{ik} }
\newcommand{\Akj}{ A_{kj} }

\newcommand{\tenidx}[9]{ (#1\!:\!#2\!:\!#3,#4\!:\!#5\!:\!#6,#7\!:\!#8\!:\!#9) }
\newcommand{\matidx}[6]{ (#1\!:\!#2\!:\!#3,#4\!:\!#5\!:\!#6) }
\newcommand{\vecidx}[3]{ (#1\!:\!#2\!:\!#3) }

%\newcommand{\vecrow}{\mathcal{V}^{1\triangleleft 0}}
%\newcommand{\veccol}{\mathcal{V}^{0\triangleleft 1}}
%\newcommand{\matcol}{\mathcal{M}^0}
%\newcommand{\matrow}{\mathcal{M}^1}
\newcommand{\vecrow}{\ensuremath{V_R}}
\newcommand{\veccol}{\ensuremath{V_C}}
\newcommand{\matcol}{\ensuremath{M_C}}
\newcommand{\matrow}{\ensuremath{M_R}}

\newcommand{\rzero}{r}
\newcommand{\rone}{c}

\newcommand{\szero}{s}
\newcommand{\sone}{t}

\newcommand{\Mzero}{\ensuremath{M_C}}
\newcommand{\Mone}{\ensuremath{M_R}}

\newcommand{\Dist}[3]{\mathcal{D}^{#1\triangleleft #2\triangleleft #3}}

\newcommand{\all}{\mathbb{*}}
% \renewcommand{\all}{\N}

%%% 
%%% \newcommand{\pOO}{ p^{(0,0)} }
%%% \newcommand{\plO}{ p^{(1,0)} }
%%% \newcommand{\pll}{ p^{(1,1)} }
%%% \newcommand{\pij}{ p^{(i,j)} }
%%% \newcommand{\pii}{ p^{(i,i)} }
%%% \newcommand{\pjj}{ p^{(j,j)} }
%%% 
%%% \newcommand{\LOO}{ L^{(0,0)}{} }
%%% \newcommand{\LlO}{ L^{(1,0)}{} }
\newcommand{\Lij}{ L_{ij}{} }
\newcommand{\Ljk}{ L_{jk}{} }
\newcommand{\Ljj}{ L_{jj}{} }
\newcommand{\Lik}{ L_{ik}{} }
%%% \newcommand{\LUOO}{ \{ L \backslash U \}^{(0,0)}{} }
%%% \newcommand{\LUll}{ \{ L \backslash U \}^{(1,1)}{} }
\newcommand{\LUjj}{ \{ L \backslash U \}^{jj}{} }
%%% \newcommand{\barLOO}{ \bar{L}^{(0,0)}{} }
%%% 
%%% \newcommand{\UOO}{ U^{(0,0)}{} }
\newcommand{\Ukk}{ U^{kk}{} }
%%% \newcommand{\barUOO}{ \bar{U}^{(0,0)}{} }
%%% \newcommand{\barLUOO}{ \{ \bar{L} \backslash \bar{U} \}^{(0,0)}{} }

%%%%%%%%%%%%%%%%%%%%

%%%%\newcommand{\AOO}{ \hat{A}_{00} }
%%%%\newcommand{\AOl}{ \hat{A}_{01} }
%%%%\newcommand{\AlO}{ \hat{A}_{10} }
%%%%\newcommand{\All}{ \hat{A}_{11} }
\newcommand{\AOO}{ B }
%%%%\newcommand{\barAOO}{ \bar{B} }
\newcommand{\AOl}{ C }
%%%%\newcommand{\barAOl}{ \bar{C} }
%%%%\newcommand{\barbarAOl}{ \bar{\bar{C}} }
\newcommand{\AlO}{ D }
%%%%\newcommand{\barAlO}{ \bar{D} }
\newcommand{\All}{ E }
%%%%\newcommand{\barAll}{ \bar{E} }
%%%%\newcommand{\barbarAll}{ \bar{\bar{E}} }

\newcommand{\FS}{\mbox{\sc FS}}

\newcommand{\unb}{\mbox{\sc \scriptsize unb}}
\newcommand{\blk}{\mbox{\sc \scriptsize blk}}

\newcommand{\lap}{\mbox{\sc \scriptsize lap}}
\newcommand{\lin}{\mbox{\sc \scriptsize lin}}

\newcommand{\LUunb}{ \LU_{\unb}}
\newcommand{\LUblk}{ \LU_{\blk}}

\newcommand{\LUlinunb}{ \LU^{\lin}_{\unb}}
\newcommand{\LUlapunb}{ \LU^{\lap}_{\unb}}
\newcommand{\LUlinblk}{ \LU^{\lin}_{\blk}}
\newcommand{\LUlapblk}{ \LU^{\lap}_{\blk}}

\newcommand{\sa}{\mbox{\sc \scriptsize sa}}

\newcommand{\LUsablk}{ \LU^{\sa}_{\blk}}
\newcommand{\LUsalinblk}{ \LU^{\sa-\lin}_{\blk}}
\newcommand{\LUsalapblk}{ \LU^{\sa-\lap}_{\blk}}

\newcommand{\FSsalinblk}{ \FS^{\sa-\lin}_{\blk}}
\newcommand{\FSlin}{ \FS^{\lin}}
\newcommand{\FSlap}{ \FS^{\lap}}

\newcommand{\OOCone}{\mbox{\sc \scriptsize OOC-1}}
\newcommand{\OOCtwo}{\mbox{\sc \scriptsize OOC-2}}
\newcommand{\OOCthree}{\mbox{\sc \scriptsize OOC-3}}
\newcommand{\OOCfour}{\mbox{\sc \scriptsize OOC-4}}

\title{Notes on Parallelizing Matrix-Matrix Multiplication}
\author{Robert van de Geijn, Martin Schatz, Tze Meng Low}

\begin{document}

\maketitle

\section{A simple implementation}

In this assigment, you start with a very simple parallel matrix-matrix 
multiplication that implements $ C = A B + C $ as a sequence of rank-k updates:
\[
C = \left( 
\begin{array}{c | c | c | c}
a_0 & a_1 & \ldots & a_{k-1}
\end{array}
\right) 
\left(
\begin{array}{c}
\hat{b}^T_0 \\
\hat{b}^T_1 \\
\vdots \\
\hat{b}^T_{k-1}
\end{array}
\right) 
=
(  (\cdots  ( ( C + a_0 \hat{b}^T_0  ) + a_1 \hat{b}^T_1 )  + \cdots ) + a_{k-1} \hat{b}^T_{k-1} )
\]

\subsection{Distribution}

Consider Figure~\ref{fig:distrC}--\ref{fig:distrB}, 
which illustrate how matrices $ C $, $ A $, and $ B $ are distributed to a $ r \times c $ mesh of (MPI) processes, when $ r = c = 3 $.
They are distributed using an elemental cyclic distribution.
            
\input distr_fig_abc

\subsection{Routine {\tt ParallelRank1}}

Since the matrix multiplication can be accomplished as a loop around
rank-1 updates, I created a routine {\tt ParallelRank1} that
implements a rank-1 update with the $ p$th column $ A $ multiplying
the $ p $th row of $ B$, updating $ C $.

Notice that this routine communicates the $ p $th column of $ A $ within
rows of nodes and the $ p $th row of $ B $ within columns of nodes to
achieve the duplication of data illustrated in
Figure~\ref{fig:distrA}.  The duplicated data are in the work arrays {\tt
work\_A} and {\tt work\_B}, respectively, and the local computation is
then performed by the call to the BLAS routine {\tt dger\_}, which
performs a rank-1 update.  Notice that {\tt dger\_} uses a Fortran
interface, meaning that data must be passed by address, which explains
some of the ugliness...

\subsection{Routine {\tt ParallelMMult}}

The routine {\tt ParallelMMult} in file {\tt ParallelMMult\_1.c} simply
implements the parallel matrix-matrix multiplication as a loop around
calls to {\tt ParallelRank1}.

\subsection{The driver routine}

The driver routine sets up the MPI environment, creates random
matrices, copies (via the utility routine {\tt
  CopyMatrixGlobalToLocal}) the local part of those random matrices into the
local arrays used for the parallel matrix-matrix multiplication, etc.
It uses utility routines to create a random matrix and to compare the
contents of two matrices.

\subsection{Executing}

Copy {\tt \~rvdg/class/CS378S13/PMMult} into a directory of your own.  
Executing {\tt make run} will compile and test (with a $ 2 \times 3 $
mesh of processes).  The reported {\tt diff} should be small.

\subsection{Comment}
When we run this on a real machine (next week), you will see that this
implementation is (weakly) scalable, but achieves very low
performance.

\section{Optimization 1}

The key is to locally call a matrix-matrix multiply rather than a
rank-1 update.  For this, observe that
\[
C = \left( 
\begin{array}{c | c | c | c}
A_0 & A_1 & \ldots & A_{K-1}
\end{array}
\right) 
\left(
\begin{array}{c}
B_0 \\
B_1 \\
\vdots \\
B_{K-1}
\end{array}
\right) 
=
(  (\cdots  ( ( C + A_0 B_0  ) + A_1 B_1 )  + \cdots ) + A_{K-1} B_{K-1} )
\]

Now, I will describe how to morph {\tt ParallelRank1} into {\tt
  ParallelRankK}, which will implement 
$ C = C + A_k B_k $, where $ A_k $ has $ b_k $ columns and $ B_k $ has
$ b_k $ rows.

\subsection{Calling sequence}

Please make the calling sequence for this new routine
\begin{verbatim}
void ParallelRankK( 
		   // global matrix dimensions m, n, k
		   int global_m, int global_n, int global_k,  
		   // the global index of the current columns of A and current rows of B 
		   // to be used for the rank-b_k update
		   int p, 
                   // the block size for this rank-k update 
		   int bk, 
		   // address of where local A, B, and C are stored, as well as their 
		   // local "leading dimensions"
		   double *local_A, int local_ldimA, 
		   double *local_B, int local_ldimB, 
		   double *local_C, int local_ldimC,
		   // communicator for the row in which this node is a member
		   MPI_Comm comm_row, 
		   // communicator for the column in which this node is a member
		   MPI_Comm comm_col )
\end{verbatim}
(Notice the new parameter {\tt bk}.)

{\bf Store this first implementation in file {\tt ParallelRankK\_2.c}.}

\subsection{From {\tt ParallelRank1} to {\tt ParallelRankK\_2}}


\paragraph*{Work arrays}
 First, you will want to make it so that {\tt work\_A} can hold $ b_k $
columns and {\tt work\_B} can hold $ b_k $ columns.  Hint:  take rows
of $ B $ but store them as columns of {\tt work\_B}!!!  This will make
your life easier.

\paragraph*{Communication}
In your first implementation, perform separate
broadcasts to communicate each of the $ b_k $ columns of $ A$ and $
b_k $ rows of $ B $ by putting a loop around the broadcasts in {\tt
  ParallelRank1}.

\paragraph*{Local computation}

Also, put a loop around the rank-1 updates (calls to {\tt dger\_}).

\subsection{From {\tt ParallelMMult\_1} to {\tt ParallelMMult\_2}}

Copy {\tt ParallelMMult\_1.c} into {\tt ParallelMMult\_2.c} (but don't
change the name or calling sequence).  
Change the routine to call {\tt ParallelRankK}.  You will have to
change the loop to increment in steps of {\tt bk}.  How do you deal
with the fact that in the last iteration you may not have a full set
of {\tt bk} columns in $ A $ (or rows in $ B $)?

Change the {\tt Makefile} and see if you get the right answer.


\section{Optimization 2}

\subsection{From {\tt ParallelRankK\_2} to {\tt ParallelRankK\_3}}

Copy from one file to a new file.

\paragraph*{Local computation}
 
Replace the loop with calls to {\tt dger\_} with a single call to the
matrix-matrix multiplication routine {\tt dgemm\_}.  (An example 
of how to call this is in the driver.  Remember: {\tt work\_B} hold
the transpose of the rows of $ B $ with which you are computing!)
 
Change the {\tt Makefile} and see if you get the right answer.

\section{Optimization 3}

\subsection{From {\tt ParallelRankK\_3} to {\tt ParallelRankK\_4}}

Copy from one file to a new file.

\paragraph*{Communication}
 
Replace the loop with calls to {\tt MPI\_Bcast} with a single call to 
an appropriate collective communication.  What else do you need to do?
 
Change the {\tt Makefile} and see if you get the right answer.

\end{document}

